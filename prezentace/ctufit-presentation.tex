%% This is the ctufit-presentation document class. It is used to produce presentations for Czech Technical University, Faculty of Information Technology students and employees.
%%
%% This is version 1.0.1, built 7. 8. 2025.
%% 
%% Get the newest version from
%% https://gitlab.fit.cvut.cz/theses-templates/FITpresentation-LaTeX/
%%
%%
%% Copyright 2025, Tomas Novacek
%%
%% This work may be distributed and/or modified under the
%% conditions of the LaTeX Project Public License, either version 1.3
%% of this license or (at your option) any later version.
%% The latest version of this license is in
%%  https://www.latex-project.org/lppl.txt
%% and version 1.3 or later is part of all distributions of LaTeX
%% version 2005/12/01 or later.
%%
%% This work has the LPPL maintenance status `maintained'.
%%
%% The current maintainer of this work is Tomas Novacek (novacto3@fit.cvut.cz).
%% Alternatively, submit bug reports to the tracker at
%% https://gitlab.fit.cvut.cz/theses-templates/FITpresentation-LaTeX/issues
%%
%%

% arara: xelatex
% arara: biber
% arara: xelatex
% arara: xelatex

%%%%%%%%%%%%%%%%%%%%%%%%%%%%%%%%%%%%%%%%%
% CLASS OPTIONS
% language: czech/english/slovak%
% nobg: removes lion background image
% nonumber: removes page numbers from the sidebar
% sage: changes layout so it matches SAGELAB screen size
% colour: color theme of the presentation - options are orange (default) or blue
%%%%%%%%%%%%%%%%%%%%%%%%%%%%%%%%%%%%%%%%%
\documentclass[czech,blue]{ctufit-presentation}

\usepackage[backend=biber]{biblatex}
\usepackage{caption} % MINE
\addbibresource{data.bib}

\ctufittitle{Zpracování a analýza záznamů z emisních kontrol vozidel v České republice}
\ctufitsubtitle{Vedoucí práce: Mgr. Adam Szabó} % Remove the command if you do not want a subtitle on the title slide.
\ctufitauthor{Autor: Adam Prokop} % Remove the command if you do not want the author on the title slide.
\ctufitinstitute{FIT ČVUT} % Remove the command if you do not want the institution on the title slide.
\ctufitdate{17. prosince 2025} % Remove the command if you do not want the date on the title slide.



\begin{document}

\begin{frame}[plain,t]
	\titlepage
\end{frame}

\begin{frame}
	\frametitle{Struktura}
	\tableofcontents
\end{frame}

%=============================================================================================

\section{Cíle práce}
\begin{frame}
	\frametitle{Cíle}
	\begin{itemize}
		\item Detekce anomálií v záznamech emisních kontrol
		\item Vytvoření modelu pravděpodobnosti úspěchu kontroly
		\item Analýza agregovaných trendů v čase
		\item Vizualizace výsledků přehlednou formou
	\end{itemize}
\end{frame}

\section{Motivace výběru tématu}
\begin{frame}
	\frametitle{Motivace}
	\begin{itemize}
		\item Přispění do veřejné diskuse
		\item Aktualizace předchozí analýzy Ministerstva dopravy
		\begin{figure}
            \includegraphics[width=\linewidth]{images/predchozi_analyza.png} 
			\captionsetup{justification=raggedright,singlelinecheck=false,font=scriptsize}
			\caption{Předchozí analýza emisí}
        \end{figure}
	\end{itemize}
\end{frame}

\section{Praktická část práce}
\subsection{Získání dat}
\begin{frame}[fragile]
	\frametitle{Praktická část}
	\framesubtitle{Získání dat}
	\begin{enumerate}
		\item Lokalizace relevantních datasetů na \href{https://data.gov.cz/}{data.gov.cz}
		\begin{figure}
			\vfill
            \includegraphics[width=\linewidth]{images/portal_o_datech.png} 
			\captionsetup{justification=raggedright,singlelinecheck=false,font=scriptsize}
			\caption{Portál o datech}
        \end{figure}
	\end{enumerate}
\end{frame}

\begin{frame}[fragile]
	\frametitle{Praktická část}
	\framesubtitle{Získání dat}
	\begin{enumerate}
		\setcounter{enumi}{1}
		\item Automatické stažení dat přes SPARQL endpoint
			\begin{itemize}
				\item Denní členění – tisíce záznamů
				\item XML formát – přes 150 GB dat
			\end{itemize}
	\end{enumerate}
	{\scriptsize
		\begin{verbatim}
			SELECT ?title ?downloadURL 
			WHERE {{
				<{parent_dateset_iri}> dcat:seriesMember ?dataset.
				?dataset dcat:distribution ?distribution.
				?dataset dcterms:title ?title.
				?distribution dcat:downloadURL ?downloadURL.
				FILTER(LANG(?title) = "cs")
			}}
		\end{verbatim}
	}
	\begin{figure}
		\includegraphics[width=\linewidth]{images/sparql_data_z_mericich_pristroju.png} 
		\captionsetup{justification=raggedright,singlelinecheck=false,font=scriptsize}
		\caption{Dokumentace datasetu ve SPARQL}
	\end{figure}
\end{frame}

\subsection{Analýza a čištění dat}
\begin{frame}[fragile]
	\frametitle{Praktická část}
	\framesubtitle{Analýza a čištění dat}
	\begin{enumerate}
		\item Extrakce informací z XML do parquet formátu
	\end{enumerate}
	{\tiny
		\begin{verbatim}
			<DatumProhlidky>2022-08-27</DatumProhlidky>
			<Stanice>
				<Cislo>3237</Cislo>
			</Stanice>
			<CasoveUdaje>
				<Zahajeni>2022-08-27T11:58:54.0870000+02:00</Zahajeni>
				<Ukonceni>2022-08-27T12:05:37.3870000+02:00</Ukonceni>
			</CasoveUdaje>
			<OdpovednaOsoba>49493</OdpovednaOsoba>
			<PristrojData>
				<prohlidka cisloProtokolu="CZ-003237-22-08-0921"
				datumProhlidky="2022-08-27T12:05:00">
					<mericiPristroj vyrobce="ACTIA CZ s.r.o." typ="AT505" verze="-"
					OBD="R+OBD" verzeSoftware="3.05.3"/>
				</prohlidka>
				<vozidlo>
					<VIN>WF0MXXGCDM6M38271</VIN>
					<tovazniZnacka>FORD</tovazniZnacka>
					<typVozidla>FOCUS C-MAX</typVozidla>
					<typMotoru>HWDA</typMotoru>
					<cisloMotoru>-</cisloMotoru>
					<stavTachometru>151057</stavTachometru>
					<rokVyroby>2006</rokVyroby>
					<datumPrvniRegistrace>2006-08-28</datumPrvniRegistrace>
					<palivo>BA</palivo>
				</vozidlo>
			. . .
		\end{verbatim}
	}
\end{frame}

\begin{frame}[fragile]
	\frametitle{Praktická část}
	\framesubtitle{Analýza a čištění dat}
	\begin{enumerate}
		\setcounter{enumi}{1}
		\item Uložení do samostatných souborů po dnech
			\begin{itemize}
				\item Nižší nároky na operační paměť
			\end{itemize}
		{\tiny
			\begin{verbatim}
				prohlidky_Prohlídky vozidel STK a SME za 01-01-2019.parquet
				prohlidky_Prohlídky vozidel STK a SME za 02-01-2019.parquet
				prohlidky_Prohlídky vozidel STK a SME za 03-01-2019.parquet
				prohlidky_Prohlídky vozidel STK a SME za 04-01-2019.parquet
				prohlidky_Prohlídky vozidel STK a SME za 05-01-2019.parquet
				prohlidky_Prohlídky vozidel STK a SME za 06-01-2019.parquet
				prohlidky_Prohlídky vozidel STK a SME za 07-01-2019.parquet
				prohlidky_Prohlídky vozidel STK a SME za 07-01-2019.parquet
				. . .
			\end{verbatim}
		}
		\item Načtení do polars
		{\scriptsize
			\begin{verbatim}
				lazy_df = pl.scan_parquet(
					"data/parquet/prohlidky/*.parquet",
					schema=prohlidky_schema
				)
			\end{verbatim}
		}
		\item Vyřešení nekonzistencí po obsahové stránce dat
		\item Rozdělení vozidel podle typu pohonu
	\end{enumerate}
\end{frame}

\subsection{Detekce anomálií}
\begin{frame}[fragile]
	\frametitle{Praktická část}
	\framesubtitle{Detekce anomálií}
	\begin{enumerate}
		\item Využití statistických metod pro detekci anomálií
			\begin{itemize}
				\item Metody nesupervizovaného učení
				\item Inspirace postupy aplikovanými v analýze MDČR
			\end{itemize}
		\item Vyhledání podezřelých hodnot na základě identity záznamů
	\end{enumerate}
	\begin{table}
		\includegraphics[width=\linewidth]{images/duplicitni_hodnoty.png} 
		\captionsetup{justification=raggedright,singlelinecheck=false,font=scriptsize}
		\caption{Předchozí diskuse na x.com}
	\end{table}
\end{frame}

\subsection{Vytvoření modelu}
\begin{frame}[fragile]
	\frametitle{Praktická část}
	\framesubtitle{Model pravděpodobnosti úspěchu}
	\begin{enumerate}
		\item Agregace informací z více datasetů na základě VIN
		\item Natrénování několika modelů a výběr nejvýkonnějšího
		\item Analýza důležitosti příznaků
	\end{enumerate}
\end{frame}

\subsection{Analýza tredů v čase}
\begin{frame}[fragile]
	\frametitle{Praktická část}
	\framesubtitle{Trendy}
	\begin{enumerate}
		\item Analýza změn vybraných ukazatelů v čase
		\item Predikce vývoje do budoucna
	\end{enumerate}
\end{frame}

\subsection{Vizualizace na webových stránkách}
\begin{frame}[fragile]
	\frametitle{Praktická část}
	\framesubtitle{Prezentace výsledků}
	\begin{enumerate}
		\item Zakomponování výsledků do existujících webových stránek
			\begin{itemize}
				\item Diplomová práce ing. Daniela Brotze
				\item Zobrazuje zajmavé infografiky související s měřením na STK
				\item Poskutuje predikci výsledku příští technické kontroly
			\end{itemize}
	\end{enumerate}
	\begin{figure}
		\includegraphics[width=\linewidth]{images/STK_portal.png}
		\captionsetup{justification=raggedright,singlelinecheck=false,font=scriptsize}
		\caption{Aktuální verze STK portálu}
	\end{figure}
\end{frame}

\section{Závěr}
\begin{frame}[fragile]
	\frametitle{Závěr}
	\begin{itemize}
		\item Automatizované zpracování a vizualizace skutečných dat
			\begin{itemize}
				\item V reálném čase
				\item Původní forma pro laiky nepoužitelná
			\end{itemize}
		\item Identifikace podezřelých stanic
		\item Lepší porozumění trendům
	\end{itemize}
\end{frame}

\begin{plainFrame}[colourbg]
	\plainFrameText{Děkuji za pozornost}
\end{plainFrame}
\end{document}